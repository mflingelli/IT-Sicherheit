\section*{Vorwort}
Die Gefahren, die im Internet herrschen, werden oft unterschätzt oder nicht wahrgenommen. Das Thema IT-Sicherheit wird im privaten Umfeld oft stiefmütterlich behandelt. Unter IT-Sicher\-heit versteht man alle Planungen, Maßnamen und Kontrollen, die dem Schutz der IT dienen.
\vspace{12pt}

Dieses Dokument soll einen kurzen Überblick über die wichtigsten Bereiche der IT-Sicherheit geben.  Desweiteren wird aufgezeigt, welche Möglichkeiten es gibt, wie man diese erkennen und man sich davor schützen kann. Diese Ratschläge betreffen sowohl den klassischen Com\-puter- als auch den Smartphone-Nutzer.
\vspace{12pt}

Der Abschnitt zur Datenschutzgrundverordnung (DVGVO) gibt einen sehr oberflächlichen Überblick über diese und kann nicht als rechtliche Beratung gesehen werden.
\vspace{12pt}

Die Präsentation wurde ursprünglich für die Laptop-Nutzer in der FF Steinheim e.V. konzipiert. Im Rahmen der firmeninternen Ausbildung der Auszubildenden Fachininformatiker wurde sie in einer früheren Version auch in der Firma \href{https://www.blackned.de}{blackned GmbH} gehalten. Die Feuerwehrversion wurde um einige Grundlagen und aktuelle Entwicklungen ergänzt.
\vspace{24pt}

Steinheim, \today \hfill Markus Flingelli
\newpage